\section{Estimativa de Fase}\label{sec:estimativa-de-fase}

A estimativa de fase é a chave principal para vários algoritmos
quânticos, como a busca de ordem e a fatoração.
Suponha o operador unitário \(U\) tenha autovetor \(\ket{u}\) com autovalor
\(e^{2\pi i \varphi}\), onde \(\varphi\) é desconhecido e descobri-lo é
o objetivo.

Para isso são necessários dois conjuntos de qubits.
O primeiro possui \(t\) qubits no estado \(\ket{0}\), onde \(t\) depende da precisão
desejada na estimativa e da probabilidade desejada da estimativa estar
correta.
Já o segundo, inicia no estado \(\ket{u}\) e possui a quantidade de qubits necessária para representar tal estado.

\[
    t = n + \left\lceil \log \left( 2 + \dfrac{1}{2\varepsilon}\right) \right\rceil
\]

Para se obter \(\varphi\) com uma precisão de \(n\) bits e probabilidade
de sucesso de pelo menos \(1 - \varepsilon\), precisa-se de

Existem três estágios durante a estimativa.
Primeiramente a porta de Hadamard é aplicada no primeiro conjunto e na sequência operações
\(U\)-controladas são aplicadas no segundo conjunto, com \(U\) elevada a
sucessivas potências de dois, conforme a \autoref{fig:primeiro-estagio}.

\begin{figure}[!htb]
    \centering
    \includesvg[width=0.75\textwidth,height=\textheight]{utils/algoritmos/phase_estimation/phase_estimation.svg}
    \caption{Primeiro estágio.}
    \label{fig:primeiro-estagio}
\end{figure}

Com isso, o estado final do primeiro conjunto é

\[
    \dfrac{1}{\sqrt{2^t}} (\ket{0} + e^{2\pi i 2^{t-1}\varphi} \ket{1}) (\ket{0} + e^{2\pi i 2^{t-2}\varphi} \ket{1}) \dots (\ket{0} + e^{2\pi i 2^{0}\varphi} \ket{1}) = \dfrac{1}{\sqrt{2^t}} \sum_{k=0}^{2^t - 1} e^{2\pi i \varphi k} \ket{k}
\]

Agora, suponha que com exatamente \(t\) bits seja possível representar
\(\varphi\) como \(0.\varphi_1 \dots \varphi_t\).
Então o estado resultante pode ser reescrito como

\[
    \dfrac{1}{\sqrt{2^t}} (\ket{0} + e^{2\pi i 0.\varphi_t} \ket{1}) (\ket{0} + e^{2\pi i 0.\varphi_{t-1}\varphi_t} \ket{1}) \dots (\ket{0} + e^{2\pi i 0.\varphi_1 \dots \varphi_t} \ket{1})
\]

No segundo estágio, a transformada de Fourier Quântica inversa é
aplicada no primeiro conjunto de qubits.
Com isso, tem-se o estado \(\ket{\varphi_1 \dots \varphi_t}\), o que resulta exatamente em
\(\varphi\).

Por fim, no terceiro estágio, é feita a leitura do primeiro conjunto de
qubits para se obter a fase \(\varphi\).

A estimativa de fase pode ser representada de forma geral pela \autoref{fig:circuito-completo}.

\begin{figure}[!htb]
    \centering
    \includesvg[width=0.75\textwidth,height=\textheight]{utils/algoritmos/phase_estimation/phase_estimation_full.svg}
    \caption{Circuito completo da Estimativa de Fase.}
    \label{fig:circuito-completo}
\end{figure}

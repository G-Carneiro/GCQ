\section{Busca de Ordem}\label{sec:busca-de-ordem}

Seja \(x\) e \(N, x < N\), sem fatores comuns, a \textbf{ordem} de \(x\)
módulo \(N\) é definida como o menor inteiro positivo \(r\) tal que
\(x^r = 1 \pmod N\).
Até o momento esse é um problema difícil de ser resolvido por um computador clássico usando recursos polinomiais nos \(O(L)\) bits necessários para especificar o problema, com \(L \equiv \lceil \log (N) \rceil\).

A busca de ordem pode ser reduzida ao problema da estimativa de fase com
algumas manipulações algébricas um tanto complexas\cite{nielsen_chuang_2010}.
Mas consiste em aplicar o operador unitário \(U\) de forma que

\[
    U \ket{y} \equiv \ket{x y \pmod N}
\]

Para a redução ser possível, duas condições devem ser cumpridas.
A primeira é que seja possível implementar uma operação
\(U^{2^j}\)-controlada, para qualquer inteiro \(j\).
Ela é satisfeita usando-se o procedimento conhecido como {exponenciação modular}.

Já a segunda é a preparação de um auto-estado com autovalor não-trivial,
o que requer saber o valor de \(r\).
Felizmente, isso é resolvido com a estimativa de fase.

\subsection{Exponenciação Modular}\label{subsec:exponenciacao-modular}

Com algumas manipulações, é perceptível a equivalência entre a aplicação
de portas \(U^{2^j}\)-controladas e o uso de {exponencial modular}, como
segue:

\[
    \begin{array}{lll}
        \ket{z}\ket{y} &\to& \ket{z} U^{z_t 2^{t-1}} \dots U^{z_1 2^0} \ket{y} \\
        &=& \ket{z} \ket{x^{z_t 2^{t-1}} \times \dots \times x^{z_1 2^0} y \pmod N} \\
        &=& \ket{z} \ket{x^z y \pmod N}
    \end{array}
\]

E essa operação pode ser facilmente realizada com a técnica de
computação reversível.

\subsection{Circuito}\label{subsec:circuito}

O circuito completo da busca de ordem é muito semelhante ao da
estimativa de fase, como mostra a \autoref{fig:busca-ordem}.

\begin{figure}[!htb]
    \centering
    \includesvg[width=0.75\textwidth,height=\textheight]{utils/algoritmos/order_finding/order_finding.svg}
    \caption{Circuito da Busca de Ordem.}
    \label{fig:busca-ordem}
\end{figure}

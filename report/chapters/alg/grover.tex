\section{Algoritmo de Busca de Grover}\label{sec:algoritmo-de-busca-de-grover}

Aqui será abordado como implementar o algoritmo quântico de Grover para
resolver o problema da busca em base de dados desordenada, o qual é mais
eficiente que as soluções clássicas.

\subsection{Problema}\label{subsec:problema}

O problema pode ser resumido em encontrar um determinado elemento dado
uma lista desordenada de tamanho \(2^n\).
Para isso, deve ser criada uma função booleana \(f: \{0, 1\}^n \to \{0, 1\}\), a qual só sinalize `1'
para o elemento desejado.

\[\begin{aligned}
f(x) =  \begin{cases}
            0, x \ne x_0 \\
            1, x = x_0
        \end{cases}
\end{aligned}\]

\textbf{Observação}

Cada elemento da lista será codificado em um estado da base
computacional \(\left| i \right\rangle, i = \{0, \dots, 2^n - 1\}\).

\subsection{Oráculo}\label{subsec:oraculo}

Um oráculo faz o mesmo que a função booleana descrita anteriormente.
Existem dois tipos de oráculos, XOR e os de fase, nesse problema será
utilizado o de fase.

\subsubsection{Oráculo de Fase}\label{subsubsec:oraculo-de-fase}

O oráculo de fase introduzirá uma fase de \(\pi\), ou seja, multiplicará
por \(-1\).

\begin{itemize}
\tightlist
\item
  Exemplo: dado
  \(\left| \psi \right\rangle = \dfrac{1}{\sqrt{2}}\left( \left| 0 \right\rangle + \left| 1 \right\rangle\right)\).
  Se o oráculo for usado para marcar o estado
  \(\left| 1 \right\rangle\), o resultado será
  \(O(\left| \psi \right\rangle) = \dfrac{1}{\sqrt{2}}\left( \left| 0 \right\rangle - \left| 1 \right\rangle\right)\).
\end{itemize}

\subsection{Algoritmo}\label{subsec:algoritmo}

O primeiro passo consiste em aplicar a porta de Hadamard em todos os
qubits para se conseguir uma superposição com pesos iguais.

\[H^{\otimes n}\left|0\right\rangle^{\otimes n}.\]

Após isso é preciso aplicar \(G\) (operador de Grover) \(k\) vezes, em que \(G\)
representa a seguinte sequência de passos:

\begin{enumerate}
\def\labelenumi{\arabic{enumi}.}
\item
  Aplicar o oráculo de fase para o estado desejado;
\item
  Aplicar a porta de Hadamard em todos os qubits;
\item
  Aplicar o operador \(2 \left| 0 \right\rangle \left\langle 0 \right| - I\);
  Na notação utilizada aqui \(\left| 0 \right\rangle\) representa \(\left| 0 \right\rangle^{\otimes n}\).
\item
  Aplicar a porta de Hadamard em todos os qubits.
\end{enumerate}

\subsubsection{Notação Auxiliar}\label{subsubsec:notacao-auxiliar}

\[\begin{aligned}
\begin{matrix}
\mathbb{B}_n: \text{conjunto de todas as palavras de n bits}. \\
\mathbb{M}: \text{conjunto de todos itens desejados}. \\
N = 2^n
\begin{cases}
n: \text{número de qubits}. \\
N: \text{número de itens}.
\end{cases} \\
M: \text{número de itens desejados (usaremos } M = 1). \\
\left| \alpha \right\rangle \coloneqq \displaystyle \sum_{\substack{x \in \mathbb{B}_n \\ f(x) = 0}} \frac{\left| x \right\rangle}{\sqrt{N - M}} \\
\left| \beta \right\rangle \coloneqq \displaystyle\sum_{\substack{x \in \mathbb{B}_n \\ f(x) = 1}} \dfrac{\left| x \right\rangle}{\sqrt{M}} : \text{itens desejados}. \\
S \coloneqq \text{span}_\mathbb{R}\{\left| \alpha \right\rangle, \left| \beta \right\rangle \} : \text{espaço vetorial gerado por } \left| \alpha \right\rangle \text{ e } \left| \beta \right\rangle.
\end{matrix}
\end{aligned}\]

\subsubsection{Difusor}\label{subsubsec:difusor}

\[\begin{aligned}
\begin{matrix}
2 \left| 0 \right\rangle \left\langle 0 \right| - I &=& 2 \cdot
\begin{bmatrix}
1       \\
0       \\
\vdots  \\
0
\end{bmatrix}_{n \times 1}
\begin{bmatrix}
1 & 0 & \cdots & 0
\end{bmatrix}_{1 \times n} -
\begin{bmatrix}
1 & 0 & \cdots & 0 \\
0 & 1 & \cdots & 0 \\
\vdots & \vdots & \ddots & 0 \\
0 & 0 & \cdots & 1
\end{bmatrix}_{n \times n}
\\
&=&
\begin{bmatrix}
2 & 0 & \cdots & 0 \\
0 & 0 & \cdots & 0 \\
\vdots & \vdots & \ddots & 0 \\
0 & 0 & \cdots & 0
\end{bmatrix}_{n \times n} -
\begin{bmatrix}
1 & 0 & \cdots & 0 \\
0 & 1 & \cdots & 0 \\
\vdots & \vdots & \ddots & 0 \\
0 & 0 & \cdots & 1
\end{bmatrix}_{n \times n}
\\
&=&
\begin{bmatrix}
1 & 0 & \cdots & 0\\
0 & -1 & \cdots & 0 \\
\vdots & \vdots & \ddots & 0 \\
0 & 0 & \cdots & -1
\end{bmatrix}_{n \times n}
\\
&=& \left| 0\dots 0 \right\rangle \left\langle 0 \dots 0 \right| - \left| 0 \dots 1 \right\rangle \left\langle 0 \dots 1 \right| -\dots - \left| 1 \dots 1\right\rangle \left\langle 1 \dots 1 \right|
\end{matrix}
\end{aligned}\]

Usando o conceito de fase global, é possível escrever o resultado de
outra forma, sendo ela:

\[-\left| 0\dots 0 \right\rangle \left\langle 0 \dots 0 \right| + \left| 0 \dots 1 \right\rangle \left\langle 0 \dots 1 \right| + \dots + \left| 1 \dots 1\right\rangle \left\langle 1 \dots 1 \right|\]

Para obter esse resultado, basta usar o oráculo de fase visto
anteriormente e usá-lo para marcar o estado \(\left| 0 \right\rangle\).

\subsubsection{Primeira Aplicação de $G$}\label{subsubsec:primeira-aplicacao-de-g}

Antes de fazer as aplicações, temos:

\[\begin{aligned}
\begin{matrix}
\left| \psi_0 \right\rangle &=& \left| + \right\rangle^{\otimes n} \\
&=& \sum_{x \in \mathbb{B}_n} \dfrac{\left| x \right\rangle}{\sqrt{N}} \\
&=& \sum_{\substack{x \in \mathbb{B}_n \\ x \ne x_0}}\dfrac{\left| x \right\rangle}{\sqrt{N}} + \dfrac{\left| x_0 \right\rangle}{\sqrt{N}} \\
&=& \dfrac{\sqrt{N - 1}}{\sqrt{N}}\sum_{\substack{x \in \mathbb{B}_n \\ x \ne x_0}}\dfrac{\left| x \right\rangle}{\sqrt{N - 1}} + \dfrac{\left| x_0 \right\rangle}{\sqrt{N}} \\
&=& \dfrac{\sqrt{N - 1}}{\sqrt{N}} \left| \alpha \right\rangle + \dfrac{1}{\sqrt{N}} \left| \beta \right\rangle
\end{matrix}
\end{aligned}\]

Após a aplicação do oráculo (\autoref{fig:oraculo}):

\[\begin{aligned}
\begin{matrix}
\left| \psi_1 \right\rangle &=& O_F \left| \psi_0 \right\rangle \\
&=& \dfrac{\sqrt{N - 1}}{\sqrt{N}} \left| \alpha \right\rangle - \dfrac{1}{\sqrt{N}} \left| \beta \right\rangle
\end{matrix}
\end{aligned}\]

\begin{figure}[!htb]
  \centering
  \includesvg[width=0.5\textwidth,height=\textheight]{utils/algoritmos/grover/oracle_application.svg}
  \caption{Aplicação do oráculo de fase equivale a uma reflexão em relação ao eixo \(\left| \alpha \right\rangle\)\cite{giovani}.}
  \label{fig:oraculo}
\end{figure}

Após a aplicação de \(2\left| \psi_0 \right\rangle \left\langle \psi_0 \right| - I\), \autoref{fig:difusor}:

\[\begin{aligned}
\begin{matrix}
\left| \psi_2 \right\rangle &=& (2\left| \psi_0 \right\rangle \left\langle \psi_0 \right| - I) \left| \psi_1 \right\rangle \\
&=& 2 \left\langle \psi_0 \mid \psi_1 \right\rangle \left| \psi_0 \right\rangle - \left| \psi_1 \right\rangle
\end{matrix}
\end{aligned}\]

\begin{figure}[!htb]
  \centering
  \includesvg[width=0.5\textwidth,height=\textheight]{utils/algoritmos/grover/diffuser.svg}
  \caption{Aplicação do operador \(2 \left| \psi_0 \right\rangle \left\langle \psi_0 \right| - I\) equivale a uma reflexão em relação à reta determinada pelo vetor \(\left| \psi_0 \right\rangle\)\cite{giovani}.
  }
  \label{fig:difusor}
\end{figure}

Com isso, após uma reflexão sobre o estado \(\left| \psi_0 \right\rangle\), tem-se o resultado da \autoref{fig:first-grover}.

\begin{figure}[!htb]
  \centering
  \includesvg[width=0.5\textwidth,height=\textheight]{utils/algoritmos/grover/first_grover.svg}
  \caption{Aplicação do operador \(G\).
  O efeito corresponde à rotação do vetor por um ângulo \(\theta\) no sentido anti-horário\cite{giovani}.
  }
  \label{fig:first-grover}
\end{figure}

\subsection{Aplicações Sucessivas de \(G\)}\label{subsec:aplicacoes-sucessivas-de-g}

A cada repetição tem-se uma rotação no sentido anti-horário, logo, o vetor estará se afastando de \(\left| \alpha\right\rangle\) e se aproximando de \(\left| \beta \right\rangle\) (item desejado), conforme a \autoref{fig:k-grover}.

\begin{figure}[!htb]
  \centering
  \includesvg[width=0.5\textwidth,height=\textheight]{utils/algoritmos/grover/k_applications.svg}
  \caption{Aplicações sucessivas do operador \(G\).}
  \label{fig:k-grover}
\end{figure}

\subsubsection{Número de Aplicações}\label{subsubsec:numero-de-aplicacoes}

O número de aplicações necessárias é dado por:

\[k = \dfrac{\pi}{4} \cdot \sqrt{\dfrac{N}{M}}\]

\subsection{Implementação}\label{subsec:implementacao}

Com a teoria explicada, é possível implementar o algoritmo de Grover usando Ket.
Aqui, cada estado da base computacional será representado na base decimal, \(\left| i \right\rangle, i = 0, \dots, 2^n - 1\).
Logo, é necessário passar um inteiro maior ou igual a 0 que indicará qual desses estados será marcado, também será preciso especificar quantos bits serão simulados.

Primeiro é preciso importar o necessário (\autoref{lst:import}).

\begin{listing}[!htb]
    \begin{minted}{python}
    from math import sqrt, pi
    from ket import *
    \end{minted}
    \caption{Importando bibliotecas.}
    \label{lst:import}
\end{listing}

Para facilitar, o algoritmo vai ser dividido em quatro partes:

\begin{enumerate}
\tightlist
\item
  Oráculo de fase;
\item
  Difusor;
\item
  Operador de Grover;
\item
  Função principal \texttt{grover()}, a qual será chamada e realizará
  todas as operações necessárias.
\end{enumerate}

\subsubsection{Oráculo de fase}\label{subsubsec:oraculo-de-fase2}

\begin{listing}[!htb]
    \begin{minted}{python}
        def phase_oracle(qubits: quant, state: int):
            with around(I if state&1 else X, qubits[-1]):
                ctrl(qubits[:-1], Z, qubits[-1], on_state=state>>1)
    \end{minted}
    \caption{Oráculo de fase.}
    \label{lst:oraculo}
\end{listing}

O que está sendo feito no \autoref{lst:oraculo} é receber os qubits e o estado desejado como argumentos e aplicando a porta \(Z\) controlada.
Se os qubits estiverem no estado \texttt{state} a porta \texttt{Z} é aplicada
no último qubit (\texttt{qubit{[}-1{]}}).
Como essa operação controlada aplica a fase -1 apenas quando os qubits estão no estado \(\left|1\right>\), a porta \texttt{flipc} é usada para permutar o estado desejado (\texttt{state}) para o estado de controle (\(\left|1\cdots1\right>\)).
Com o \texttt{with\ around} é garantido que a permutação será desfeita após a aplicação da porta controlada.

\subsubsection{Difusor}\label{subsubsec:difusor2}

O difusor (\autoref{lst:difusor}) é muito semelhante ao código do oráculo, porém nele a porta de hadamard é aplicada antes e depois da porta controlada, e o estado a ser marcado é \(\left| 0 \right\rangle\).

\begin{listing}[!htb]
    \begin{minted}{python}
        def grover_diffuser(qubits: quant):
            with around(H, qubits):
                phase_oracle(qubits, 0)
    \end{minted}
    \caption{Difusor}
    \label{lst:difusor}
\end{listing}

\subsubsection{Operador de Grover}\label{subsubsec:operador-de-grover}

Agora partindo para o operador de Grover, tem-se o \autoref{lst:grover}, consistindo apenas da chamada do oráculo e depois do difusor.

\begin{listing}[!htb]
    \begin{minted}{python}
        def grover_operator(qubits: quant, state: int):
            phase_oracle(qubits, state)
            grover_diffuser(qubits)
    \end{minted}
    \caption{Operador de Grover.}
    \label{lst:grover}
\end{listing}


\subsubsection{Função Principal}\label{subsubsec:funcao-principal}

Para finalizar, a função principal, a qual irá criar os qubits e definir quantas vezes será preciso aplicar o operador de Grover.

\begin{listing}[!htb]
    \begin{minted}{python}
        def grover(state: int, num_qubits: int) -> int:
            qubits = quant(num_qubits)
            entries = 2**num_qubits
            steps = int((pi / 4) * sqrt(entries))

            H(qubits)

            for _ in range(steps):
                grover_operator(qubits, state)

            return measure(qubits).value
    \end{minted}
    \caption{Função principal do algoritmo.}
    \label{lst:principal}
\end{listing}

Após todas as operações, é feito a medida e retorna-se o valor obtido.
Como se trata de um algoritmo probabilístico, nem sempre o estado desejado será o resultado, porém é esperado que tenha alta taxa de sucesso.
Com tudo pronto pode-se chamar a função \texttt{grover()}.

\subsubsection{Saída}\label{subsubsec:saida}

O \autoref{lst:exemplo} foi criado para testar o funcionamento do algoritmo e se obteve a saída presente no \autoref{lst:saida}.

\begin{listing}[!htb]
    \begin{minted}{python}
        from random import randrange
        num_qubits = 10
        for state in [randrange(2**num_qubits) for _ in range(10)]:
            print('Procurando estado', state, '...', end=' ')
            print('Estado medido:', grover(state, num_qubits))
    \end{minted}
    \caption{Função principal do algoritmo.}
    \label{lst:exemplo}
\end{listing}


\begin{listing}[!htb]
    \caption{Saída dos testes.}
    \label{lst:saida}
    \begin{minted}{python}
        Procurando estado 882 ... Estado medido: 882
        Procurando estado 1013 ... Estado medido: 1013
        Procurando estado 557 ... Estado medido: 557
        Procurando estado 490 ... Estado medido: 490
        Procurando estado 163 ... Estado medido: 163
        Procurando estado 429 ... Estado medido: 429
        Procurando estado 300 ... Estado medido: 300
        Procurando estado 184 ... Estado medido: 184
        Procurando estado 394 ... Estado medido: 394
        Procurando estado 763 ... Estado medido: 763
    \end{minted}
\end{listing}

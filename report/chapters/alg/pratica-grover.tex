\subsection{Implementação}\label{subsec:implementacao}

Com a teoria explicada, é possível implementar o algoritmo de Grover usando Ket.
Aqui, cada estado da base computacional será representado na base decimal, \(\left| i \right\rangle, i = 0, \dots, 2^n - 1\).
Logo, é necessário passar um inteiro maior ou igual a 0 que indicará qual desses estados será marcado, também será preciso especificar quantos bits serão simulados.

Primeiro é preciso importar o necessário (\autoref{lst:import}).

\begin{listing}[!htb]
    \begin{minted}{python}
    from math import sqrt, pi
    from ket import *
    \end{minted}
    \caption{Importando bibliotecas.}
    \label{lst:import}
\end{listing}

Para facilitar, o algoritmo vai ser dividido em quatro partes:

\begin{enumerate}
\tightlist
\item
  Oráculo de fase;
\item
  Difusor;
\item
  Operador de Grover;
\item
  Função principal \texttt{grover()}, a qual será chamada e realizará
  todas as operações necessárias.
\end{enumerate}

\subsubsection{Oráculo de fase}\label{subsubsec:oraculo-de-fase2}

\begin{listing}[!htb]
    \begin{minted}{python}
        def phase_oracle(qubits: quant, state: int):
            with around(I if state&1 else X, qubits[-1]):
                ctrl(qubits[:-1], Z, qubits[-1], on_state=state>>1)
    \end{minted}
    \caption{Oráculo de fase.}
    \label{lst:oraculo}
\end{listing}

O que está sendo feito no \autoref{lst:oraculo} é receber os qubits e o estado desejado como argumentos e aplicando a porta \(Z\) controlada.
Se os qubits estiverem no estado \texttt{state} a porta \texttt{Z} é aplicada
no último qubit (\texttt{qubit{[}-1{]}}).
Como essa operação controlada aplica a fase -1 apenas quando os qubits estão no estado \(\left|1\right>\), a porta \texttt{flipc} é usada para permutar o estado desejado (\texttt{state}) para o estado de controle (\(\left|1\cdots1\right>\)).
Com o \texttt{with\ around} é garantido que a permutação será desfeita após a aplicação da porta controlada.

\subsubsection{Difusor}\label{subsubsec:difusor2}

O difusor (\autoref{lst:difusor}) é muito semelhante ao código do oráculo, porém nele a porta de hadamard é aplicada antes e depois da porta controlada, e o estado a ser marcado é \(\left| 0 \right\rangle\).

\begin{listing}[!htb]
    \begin{minted}{python}
        def grover_diffuser(qubits: quant):
            with around(H, qubits):
                phase_oracle(qubits, 0)
    \end{minted}
    \caption{Difusor}
    \label{lst:difusor}
\end{listing}

\subsubsection{Operador de Grover}\label{subsubsec:operador-de-grover}

Agora partindo para o operador de Grover, tem-se o \autoref{lst:grover}, consistindo apenas da chamada do oráculo e depois do difusor.

\begin{listing}[!htb]
    \begin{minted}{python}
        def grover_operator(qubits: quant, state: int):
            phase_oracle(qubits, state)
            grover_diffuser(qubits)
    \end{minted}
    \caption{Operador de Grover.}
    \label{lst:grover}
\end{listing}


\subsubsection{Função Principal}\label{subsubsec:funcao-principal}

Para finalizar, a função principal, a qual irá criar os qubits e definir quantas vezes será preciso aplicar o operador de Grover.

\begin{listing}[!htb]
    \begin{minted}{python}
        def grover(state: int, num_qubits: int) -> int:
            qubits = quant(num_qubits)
            entries = 2**num_qubits
            steps = int((pi / 4) * sqrt(entries))

            H(qubits)

            for _ in range(steps):
                grover_operator(qubits, state)

            return measure(qubits).value
    \end{minted}
    \caption{Função principal do algoritmo.}
    \label{lst:principal}
\end{listing}

Após todas as operações, é feito a medida e retorna-se o valor obtido.
Como se trata de um algoritmo probabilístico, nem sempre o estado desejado será o resultado, porém é esperado que tenha alta taxa de sucesso.
Com tudo pronto pode-se chamar a função \texttt{grover()}.

\subsubsection{Saída}\label{subsubsec:saida}

O \autoref{lst:exemplo} foi criado para testar o funcionamento do algoritmo e se obteve a saída presente no \autoref{lst:saida}.

\begin{listing}[!htb]
    \begin{minted}{python}
        from random import randrange
        num_qubits = 10
        for state in [randrange(2**num_qubits) for _ in range(10)]:
            print('Procurando estado', state, '...', end=' ')
            print('Estado medido:', grover(state, num_qubits))
    \end{minted}
    \caption{Função principal do algoritmo.}
    \label{lst:exemplo}
\end{listing}


\begin{listing}[!htb]
    \caption{Saída dos testes.}
    \label{lst:saida}
    \begin{minted}{python}
        Procurando estado 882 ... Estado medido: 882
        Procurando estado 1013 ... Estado medido: 1013
        Procurando estado 557 ... Estado medido: 557
        Procurando estado 490 ... Estado medido: 490
        Procurando estado 163 ... Estado medido: 163
        Procurando estado 429 ... Estado medido: 429
        Procurando estado 300 ... Estado medido: 300
        Procurando estado 184 ... Estado medido: 184
        Procurando estado 394 ... Estado medido: 394
        Procurando estado 763 ... Estado medido: 763
    \end{minted}
\end{listing}

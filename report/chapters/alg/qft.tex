\section{Transformada de Fourier Quântica}\label{sec:transformada-de-fourier-quantica}

A transformada de Fourier quântica é elemento central na resolução de
diversos algoritmos quânticos, como, por exemplo, a fatoração quântica e
a estimativa de fase.
Além disso, ela também possibilita um ganho exponencial.

\subsection{Transformada Discreta (Clássica)}\label{subsec:transformada-discreta}

Nessa notação a transformada recebe como entrada um vetor de números
complexos, \(x_0, \dots, x_{N - 1}\), onde \(N\) é fixo.
Como saída têm-se os dados transformados, ou seja, um vetor de números complexos,
\(y_0, \dots, y_{N - 1}\), definidos por

\[
    y_k \equiv \dfrac{1}{\sqrt{N}} \sum_{j=0}^{N-1} x_j e^{2\pi i jk / N}
\]

\subsection{Transformada Quântica}\label{subsec:transformada-quantica}

A transformada quântica é a mesma transformação, porém com uma notação diferente.

\[
    \ket{j} \to \dfrac{1}{\sqrt{N}} \sum_{k=0}^{N-1} e^{2 \pi  i jk /N} \ket{k}
\]

\textbf{Exemplos}

\[
    \begin{array}{ccl}
        \ket{0} &\to& \dfrac{1}{\sqrt{2}} \left( e^{\pi i\cdot 0\cdot 0} \ket{0} + e^{\pi i\cdot 0\cdot 1} \ket{1}\right) \\
        &=& \dfrac{1}{\sqrt{2}} \left( \ket{0} + \ket{1}\right) \\ \\
        \ket{1} &\to& \dfrac{1}{\sqrt{2}} \left( e^{\pi i\cdot 1\cdot 0} \ket{0} + e^{\pi i\cdot 1\cdot 1} \ket{1}\right) \\
        &=& \dfrac{1}{\sqrt{2}} \left( \ket{0} - \ket{1}\right) \\ \\
        \ket{0 \dots 0} &\to& \dfrac{1}{\sqrt{N}} \left( \ket{0 \dots 0} +  \ket{0 \dots 1} + \dots + \ket{1 \dots 1}\right) \\
        &=& \dfrac{1}{\sqrt{N}} \left( \ket{0 \dots 0} + \ket{0 \dots 1} + \dots + \ket{1 \dots 1} \right) \\ \\
        \ket{1 \dots 1} &\to& \dfrac{1}{\sqrt{N}} \left( \ket{0 \dots 0} + e^{2\pi i(N-1)/N} \ket{0 \dots 1} + \dots + e^{2\pi i(N-1)(N-1)/N} \ket{1 \dots 1}\right) \\
        &=& \dfrac{1}{\sqrt{N}} \left( \ket{0 \dots 0} + e^{2\pi i(N-1)/N} \ket{0 \dots 1} + \dots + e^{2\pi i(N^2-2N+1)/N} \ket{1 \dots 1} \right)
    \end{array}
\]

\subsubsection{Estado Arbitrário}\label{subsubsec:estado-arbitrario}

A notação anterior é aplicável para uma base ortonormal, em um estado
arbitrário possui a seguinte ação

\[
    \sum_{j=0}^{N-1} x_j \ket{j} \to \sum_{k=0}^{N-1} y_k \ket{k} \\
\]

onde cada amplitude \(y_k\) é a transformada de Fourier discreta da
amplitude \(x_j\).

\subsubsection{Representação de Produto}\label{subsubsec:representacao-de-produto}

A transformada de Fourier quântica pode ser reescrita de forma a usar
cada qubit de forma direta na fórmula.
Provavelmente essa é forma de representação mais útil da transformada, visto que a partir dela um
circuito quântico pode ser facilmente construído.

\[
    \ket{j_1 \dots j_n} \to \dfrac{1}{\sqrt{2^n}} \left( \ket{0} + e^{2 \pi i 0.j_n}\ket{1} \right) \left( \ket{0} + e^{2 \pi i 0.j_{n-1}j_n}\ket{1} \right) \dots \left( \ket{0} + e^{2 \pi i 0.j_1 \dots j_n}\ket{1} \right)
\]

\textbf{Exemplos}

\[
    \begin{array}{ccl}
        \ket{0} &\to& \dfrac{1}{\sqrt{2}} \left( \ket{0} + e^{2\pi i0.0_{\text{bin}}} \ket{1} \right) \\
        &=& \dfrac{1}{\sqrt{2}} \left( \ket{0} + \ket{1} \right) \\ \\
        \ket{1} &\to& \dfrac{1}{\sqrt{2}} \left( \ket{0} + e^{2\pi i0.1_{\text{bin}}} \ket{1} \right) \\
        \ket{1} &\to& \dfrac{1}{\sqrt{2}} \left( \ket{0} + e^{2\pi i/2} \ket{1} \right) \\
        &=& \dfrac{1}{\sqrt{2}} \left( \ket{0} - \ket{1} \right) \\ \\
        \ket{00} &\to& \dfrac{1}{\sqrt{4}} \left( \ket{0} + e^{2\pi i0.0_{\text{bin}}} \ket{1} \right) \left( \ket{0} + e^{2\pi i0.00_{\text{bin}}} \ket{1} \right) \\
        &=& \dfrac{1}{\sqrt{4}} \left( \ket{0} + \ket{1} \right) \left( \ket{0} + \ket{1} \right) \\
        &=& \dfrac{1}{\sqrt{4}} \left( \ket{00} + \ket{01} + \ket{10} + \ket{11} \right) \\ \\
        \ket{11} &\to& \dfrac{1}{\sqrt{4}} \left( \ket{0} + e^{2\pi i0.1_{\text{bin}}} \ket{1} \right) \left( \ket{0} + e^{2\pi i0.11_{\text{bin}}} \ket{1} \right) \\
        &=& \dfrac{1}{\sqrt{4}} \left( \ket{0} + e^{2\pi i/2} \ket{1} \right) \left( \ket{0} + e^{2\pi i\cdot 0.75} \ket{1} \right) \\
        &=& \dfrac{1}{\sqrt{4}} \left( \ket{0} - \ket{1} \right) \left( \ket{0} - i\ket{1} \right) \\
        &=& \dfrac{1}{\sqrt{4}} \left( \ket{00} - i\ket{01} - \ket{10} + i\ket{11} \right)
    \end{array}
\]

\subsubsection{Circuito}\label{subsubsec:circuito}

O circuito, \autoref{fig:qft}, a seguir foi feito a partir da representação de produto,
porém é necessário aplicar portas \textit{swap} ao final caso se deseje o
resultado exatamente nos mesmos qubits.

\begin{figure}[!htb]
    \centering
    \includesvg[width=\textwidth,height=\textheight]{utils/algoritmos/qft/qft.svg}
    \caption{Circuito da Transformada de Fourier Quântica.}
    \label{fig:qft}
\end{figure}

onde

\[
    R_k \equiv
    \begin{bmatrix}
        1 & 0 \\
        0 & e^{2\pi i / 2^k}
    \end{bmatrix}
\]

\subsubsection{Equivalência}\label{subsubsec:equivalencia}

Agora, para provar a equivalência entre as representações, é necessário fazer algumas manipulações algébricas na definição e chegar na representação de produto.

\[
    \begin{array}{lll}
        \ket{j} &\to& \displaystyle\dfrac{1}{2^{n/2}} \sum_{k=0}^{2^n-1} e^{2\pi ijk/2^n} \ket{k} \\ \\
        &=& \displaystyle\dfrac{1}{2^{n/2}} \sum_{k_1=0}^{1} \dots \sum_{k_n=0}^{1} e^{2\pi ij \left( \sum_{l=1}^{n}k_l 2^{-l} \right)} \ket{k_1 \dots k_n} \\ \\
        &=& \displaystyle\dfrac{1}{2^{n/2}} \sum_{k_1=0}^{1} \dots \sum_{k_n=0}^{1} \bigotimes_{l=1}^{n} e^{2\pi ij k_l 2^{-l}} \ket{k_l} \\ \\
        &=& \displaystyle\dfrac{1}{2^{n/2}} \bigotimes_{l=1}^{n} \left[ \sum_{k_l=0}^{1} e^{2\pi ij k_l 2^{-l}} \ket{k_l} \right] \\ \\
        &=& \displaystyle\dfrac{1}{2^{n/2}} \bigotimes_{l=1}^{n} \left[ \ket{0} + e^{2\pi ij 2^{-l}} \ket{1} \right] \\ \\
        &=& \dfrac{\left( \ket{0} + e^{2\pi i 0.j_n} \ket{1} \right) \left( \ket{0} + e^{2\pi i 0.j_{n-1}j_n} \ket{1} \right) \dots \left( \ket{0} + e^{2\pi i 0.j_1 j_2 \dots j_n} \ket{1} \right)}{2^{n/2}}
    \end{array}
\]
\chapter{Desenvolvimento}\label{ch:desenvolvimento}

\section{Qiskit}\label{sec:qiskit}

Inicialmente, foi planejado utilizar o material didático fornecido pelo Qiskit para o aprendizado de conceitos básicos.
Qiskit é uma ferramenta de desenvolvimento quântico criado pela IBM que permite o uso de simuladores e computadores quânticos no nível de pulsos, circuitos e outros tipos de aplicações.
Atualmente, é a ferramenta mais popular.

Com o decorrer dos estudos, alguns problemas foram identificados.
O primeiro foi referente aos cursos introdutórios, os quais não apresentavam aprofundamento necessário para se compreender os assuntos, em geral, faltavam mais explicações sobre conceitos físicos e matemáticos por trás.
Esse problema começou a ser parcialmente resolvido com a atualização dos cursos no final de 2021, mas as mudanças chegavam a passos lentos, muitas vezes as alterações feitas já estavam defasadas em relação ao desenvolvimento atual da ferramenta.
Isso se deve ao fato da ferramenta ser de código aberto e possuir uma comunidade muito ativa, recebendo várias alterações e melhorias com frequência.
Além disso, por se tratar de uma tecnologia relativamente nova e devido a essas frequentes atualizações, a ferramenta não possui um longo período de estabilidade entre versões, algo que torna códigos obsoletos rapidamente.

Outro problema encontrado foi com o acesso aos computadores quânticos da IBM.
Mesmo com a licença de estudante, os computadores disponíveis são poucos e possuem poucos qubits, tornando inviável usá-los para experimentos que exijam mais qubits.
Recentemente, a IBM retirou o computador que possuía mais qubits da licença de estudantes, dificultando ainda mais os estudos.
A alternativa, portanto, foi usar os simuladores quânticos disponíveis, mas logo se tornou enviável devido às constantes atualizações.

\section{Material Básico}\label{sec:material-basico}

Após os problemas enfrentados com o material do Qiskit, passou-se a usar\cite{giovani} como fonte de estudos.
Esse material foi criado por \citeauthor{giovani} durante seu TCC em Engenharia Eletrônica e facilitou, e muito, o aprendizado.
Ao ler o material passa-se por uma revisão de algebra linear, com foco em computação quântica, para então se ter uma introdução à mecânica quântica e computação quântica, algo que os tutoriais do Qiskit não fornecem.

\section{Tópicos Avançados}\label{sec:topicos-avancados}

Após consolidar uma base suficientemente forte com a leitura de\cite{giovani}, os livros\cite{nielsen_chuang_2010, thomas-wong} foram usados para aprofundar o conhecimento em tópicos mais avançados.
O primeiro é um clássico no assunto, mostrando conceitos fundamentais, funcionamento de computadores quânticos, principais algoritmos quânticos e mais.
Já o segundo é mais recente, trazendo explicação menos densa do conteúdo ele passa por conceitos fundamentais da computação clássica e trazendo alguns comparativos entre as diferentes formas de computação.

\section{Ket}\label{sec:ket}

Ket é uma linguagem de programação embarcada em Python, projetada para facilitar o desenvolvimento de aplicações híbridas clássica-quântica.
Ket é um projeto de código aberto derivado do trabalho de mestrado de \citeauthor{ket} no Programa de Pós-Graduação em Ciência da Computação (PPGCC) da UFSC.

Essa linguagem fornece uma grande facilidade na implementação de portas lógicas quânticas, principalmente nas controladas.
Fazendo uma comparação com o Qiskit, ela é mais estável, porém não possui tantos recursos, visto que apenas o criador desenvolve a linguagem.

\section{QuBOX}\label{sec:qubox}

Com o conteúdo aprendido, foi possível participar do projeto de extensão “Simulador quântico portátil para a linguagem de programação Ket com fins educacionais e de pesquisa”, registrado no SIGPEX sob o número 202123813 e feito em uma parceria entre a startup Quantuloop e o Grupo de Computação Quântica — UFSC.
O projeto tem o objetivo de promover a computação quântica e auxiliar no desenvolvimento de novos algoritmos, métodos e aplicações quânticas.
O público alvo são alunos e professores da educação básica ao ensino superior interessados em pesquisar, aprender e ensinar computação quântica.
Além de todos os interessados em computação quântica.

Através desse projeto é possível programar em um simulador quântico de até 30 qubits de maneira gratuita, através da linguagem \href{https://quantumket.org/}{Ket} e do simulador \href{https://qubox.ufsc.br/qubox.html}{QuBOX}.
Para o projeto, foi feito um resumo sobre \href{https://qubox.ufsc.br/qc.html}{Computação Quântica} e informações e tutoriais sobre \href{https://qubox.ufsc.br/algoritmos/index.html}{algoritmos quânticos}.

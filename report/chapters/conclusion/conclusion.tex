\chapter{Conclusão}\label{ch:conclusao}

A computação quântica é uma área ainda em crescimento, mas já apresenta grandes impactos.
As tecnologias como a criptografia precisarão se atualizar para prevenir que sistemas sejam invadidos ou quebrados pelo algoritmo de fatoração quântica.
Problemas até o momento impensáveis de serem resolvidos começaram a ganhar ótimas soluções quânticas.

Linguagens para programação de algoritmos quânticos ainda precisam amadurecer, mesmo estando evoluindo a passos largos.
Adicionalmente, computadores quânticos necessitam suporte a mais qubits em seu hardware.
É vital que o acesso a essas tecnologias seja facilitado por empresas e universidades pelo mundo, para conseguir mais profissionais capacitados na área quanto antes.

Durante a bolsa de iniciação científica, foi possível estudar conceitos básicos de computação quântica, bem como os principais algoritmos quânticos.
Com esses conhecimentos, foi possível participar um projeto de extensão (\nameref{sec:qubox}) e realizar a implementação de alguns dos algoritmos, além de fornecer uma base para o novo modo de computação.
Futuramente, quando a tecnologia quântica for mais requisitada, esses conhecimentos podem trazer uma vantajem no meio profissional, visto que ainda são poucos os que estão tendo acesso a essas informações.

Por fim, é possível utilizar o conteúdo visto na realização de uma futura pós-graduação, buscando soluções quânticas para algum problema de difícil resolução na computação clássica.
Então, a bolsa trouxe grandes benefícios para o aprendizado e perspectivas futuras.
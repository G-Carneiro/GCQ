\chapter{Computação Quântica}\label{ch:computacao-quantica}

\section{Breve Histórico}\label{sec:breve-historico}

No início do século XX os cientistas enfrentavam um grande problema, que
era explicar o comportamento da radiação emitida por um corpo negro.
A solução desse problema levou ao surgimento da Mecânica Quântica, que
segundo a hipótese de Max Planck ``a radiação só pode ser emitida ou
absorvida por um corpo negro em quantidades múltiplas inteiras de
\(hf\)'', em que \(h \approx 6,62 \cdot 10^{-34} J \cdot s\) é a
constante de Planck e \(f\) é a frequência de radiação.
A quantização da energia e de outras grandezas na escala atômica foi importante para
explicar uma série de outros fenômenos, como por exemplo o efeito
fotoelétrico e o espectro da radiação emitida por átomos e moléculas.
O desenvolvimento da Mecânica Quântica nos permitiu compreender melhor o
comportamento da matéria na escala microscópica, nesse caso particular,
os materiais semicondutores, que permitiram a criação do transistor.
Os transistores substituíram as válvulas usadas nos primeiros computadores
digitais a partir de 1955.
É importante notar que a lógica usada para realizar as operações computacionais nos nossos notebooks, PCs, tablets e smartphones é a lógica booleana, ou seja, uma lógica clássica que
envolve operações como AND, OR e NOT sobre os bits 0 e 1.
Devido a sua grande capacidade de cálculo e armazenamento, os computadores são
fundamentais para o desenvolvimento de qualquer sociedade moderna.
Essa é a chamada primeira revolução quântica.

Um dos grandes impulsionadores da computação quântica foi o físico
Richard Feynman, que no início da década de 80 sugeriu o uso de
computadores quânticos para simular sistemas quânticos.
A percepção de Feynman baseia-se no fato de que o número de configurações possíveis nos
sistemas quânticos cresce de maneira exponencial com o número de entes
(spins, elétrons, átomos, \ldots) considerados, tornando-se proibitivo para
a memória dos computadores atuais guardar tanta informação mesmo para um
número pequeno (\(< 100\)) de partículas.

Na década seguinte, os primeiros algoritmos quânticos começaram a
surgir, dentre eles, os que mais se destacaram foram o algoritmo de
busca de Grover e o de fatoração de Shor.
Este último algoritmo foi provavelmente um dos grandes responsáveis pelo desenvolvimento da
computação quântica, já que é capaz de encontrar os fatores primos \(p\)
e \(q\) que multiplicados resultavam em um número inteiro
\(N = p \cdot q\) em uma escala de tempo que cresce polinomialmente com
o tamanho do número \(N\).
Ou seja, a base do sistema de segurança RSA, amplamente usado para realizar transações bancárias no mundo todo, pode estar comprometida a partir da existência de computadores quânticos de
larga escala.

A partir da segunda década do século XXI, não apenas a computação
quântica, mas outras áreas como criptografia quântica, sensores
quânticos e simulação quântica tem recebido forte atenção não apenas do
setor acadêmico, mas também do setor industrial.
Dessa forma, tem-se observado o rápido desenvolvimento das chamadas tecnologias quânticas, o
que configura a segunda revolução quântica, uma vez que a lógica por
trás dos processos é de natureza quântica.

\section{Desenvolvimento Atual}\label{sec:desenvolvimento-atual}

A Computação Quântica ainda está em fase de amadurecimento, mas já
mostra o seu grande potencial para resolver problemas práticos, além do
algoritmo de fatoração de Shor e de busca de Grover, tais como problemas
de otimização, machine learning, logística, química quântica, finanças,
álgebra linear, entre outros.
Nos últimos cinco anos, grandes empresas como Google, IBM, Amazon e Microsoft intensificaram ainda mais os seus investimentos no setor, além de vários governos de diversos países da
América do Norte, Europa, Oceania e Ásia.
Um dos grande temores em relação ao computadores quânticos está relacionado à segurança da
informação, que afeta não apenas as transações bancárias, mas toda e
qualquer transmissão de informação sigilosa, incluindo a militar,
naturalmente.
Tentativas de barrar possíveis ataques por computadores
quânticos incluem a criptografia pós-quântica, que apesar do nome, se
baseia em métodos criptográficos clássicos.

Para quem estiver interessado em aprender computação quântica, vale
lembrar que algumas empresas disponibilizam computadores quânticos
reais, simuladores, kits e linguagens para desenvolvimento de algoritmos
ao público geral.
Por exemplo, é possível acessar gratuitamente o kit de desenvolvimento quântico criado pela IBM
(\href{https://qiskit.org/}{Qiskit}) e rodar algoritmos em alguns dos
seus computadores quânticos com poucos qubits.

\section{O que é um qubit?}\label{sec:qubit}

O qubit (\textbf{qu}antum + \textbf{bit}) é um bit quântico.
O bit clássico sempre está em um dos possíveis estados 0 ou 1, já um qubit
pode estar em ambas configurações simultaneamente.
Chamamos esse fenômeno de {superposição}.
Para representar um qubit utilizamos a notação Dirac ou “braket”:

\[\begin{aligned}
\left | \psi \right \rangle = \begin{bmatrix} a \\ b \end{bmatrix} = a \left| 0 \right \rangle + b \left| 1 \right \rangle
\end{aligned}\]

em que \(a\) e \(b\) são {amplitudes de probabilidade} (números
complexos), de modo que

\begin{gather*}
    |a|^2 \text{ representa a probabilidade de após uma medida encontrar o sistema no estado}
\left| 0 \right\rangle\\
    |b|^2 \text{ representa a probabilidade de após uma medida encontrar o sistema no estado}
\left| 1 \right\rangle\\
\end{gather*}

Como a probabilidade total deve somar \(100\%\), temos que a {condição
de normalização} para o estado \(\left| \psi \right\rangle\) é
\(|a|^2 + |b|^2 = 1\).

\subsection{Esfera de Bloch}\label{subsec:esfera-de-bloch}

Os estados de um qubit podem ser representados por meio de pontos em uma
superfície esférica de raio unitário, utilizando o sistema de
coordenadas esféricas.
Para isso, é preciso parametrizar o estado do qubit
\(\left| \psi \right\rangle = a \left| 0 \right\rangle + b \left| 1 \right\rangle\)
da seguinte forma

\[\left| \psi \right\rangle = \cos\left(  \dfrac{\theta}{2} \right) \left| 0 \right\rangle + e^{i\phi} \sin\left( \dfrac{\theta}{2} \right) \left| 1 \right\rangle \text{ tal que } \theta \in [0, \pi], \phi \in [0, 2\pi)\]

Agora, utilizando \(\theta\) e \(\phi\) no sistemas de coordenadas
esféricas, tem-se a Esfera de Bloch.
Todos os estados acessíveis a um qubit podem ser representados utilizando-se \autoref{fig:esfera-bloch}.

\begin{figure}[!htp]
    \centering
    \includesvg[width=0.8\textwidth,height=\textheight]{utils/Bloch_Sphere.svg}
    \caption{Representação de um qubit na Esfera de Bloch.}
    \label{fig:esfera-bloch}
\end{figure}

\subsection{Representação de 2 ou mais qubits}\label{subsec:repr}

Existem diversas formas de se representar um sistema de 2 qubits, seguem
algumas equivalências:

\[\left| \psi_0 \right\rangle \otimes \left| \psi_1 \right\rangle = \left| \psi_0 \right\rangle \left| \psi_1 \right\rangle = \left| \psi_0 \psi_1 \right\rangle\]

em que \(\otimes\) é produto tensorial de \(\psi_0\) com \(\psi_1\).
Seja

\[\begin{aligned}
\left| \psi_0 \right\rangle \otimes \left| \psi_1 \right\rangle
= \begin{bmatrix} a_0 \\ a_1 \end{bmatrix} \otimes \begin{bmatrix} b_0 \\ b_1 \end{bmatrix}
= \begin{bmatrix} a_0 b_0 \\ a_0 b_1 \\ a_1 b_0 \\ a_1 b_1 \end{bmatrix}
\end{aligned}\]

De forma análoga, é possível representar sistemas de \(n\) qubits como

\[\left| \psi_0 \right\rangle \otimes \left| \psi_1 \right\rangle \otimes \dots \otimes \left| \psi_n \right\rangle
= \left| \psi_0 \right\rangle \left| \psi_1 \right\rangle \dots \left| \psi_n \right\rangle
= \left| \psi_0 \psi_1 \dots \psi_n \right\rangle\]

Como será mostrado na \autoref{sec:emaranhamento}, a superposição de estados desse tipo pode levar ao emaranhamento.

\section{Etapas de um Algoritmo Quântico}\label{sec:etapas-quanticas}

\begin{figure}[!htp]
    \centering
    \includesvg[width=1\textwidth,height=\textheight]{utils/diagrama.svg}
    \caption{Etapas básicas de um algoritmo quântico}
    \label{fig:etapas-alg}
\end{figure}

De forma geral, é possível separar um algoritmo quântico em quatro etapas, como mostra a \autoref{fig:etapas-alg}.

\begin{enumerate}
\tightlist
\item
  \textbf{Preparação}: aqui cada qubit é inicializado em algum estado,
  geralmente em \(\left| 0 \right\rangle\).
\item
  \textbf{Evolução}: nessa parte o algoritmo é de fato aplicado, através
  das portas lógicas quânticas.
\item
  \textbf{Medida}: após a aplicação das portas, é necessário medir os
  qubits, para se ter o resultado do circuito.
\item
  \textbf{Pós-processamento}: finalmente, nessa etapa o resultado obtido
  deve ser interpretado de acordo com o contexto.
\end{enumerate}

\section{Comparação com Computação Clássica}\label{sec:compare}

\subsection{Entradas e Saídas}\label{subsec:entradas-e-saidas}

\begin{itemize}
\tightlist
\item
  \textbf{Clássica}: portas podem ter diferentes números de bits
  entrando e saindo.
\end{itemize}

\textbf{Exemplo}

A porta AND possui dois ou mais bits de entrada e apenas um de saída.

\begin{figure}
    \centering
    \includesvg[width=0.15\textwidth,height=\textheight]{utils/gates/and.svg}
    \caption{Representação da porta AND.}
    \label{fig:and}
\end{figure}

\begin{itemize}
\tightlist
\item
  \textbf{Quântica}: portas possuem mesmo número de qubits na entrada e
  na saída.
\end{itemize}

\subsection{Reversibilidade}\label{subsec:reversibilidade}

\begin{itemize}
\tightlist
\item
  \textbf{Clássica}: a maioria das portas clássicas não são reversíveis,
  isto é, dado uma saída não conseguimos identificar quais foram as
  entradas.
\end{itemize}

\textbf{Exemplo}

Na porta OR de dois bits podemos obter 1 como saída em três casos.

\[\begin{aligned}
\begin{array}{cc|c}
    X & Y & X \text{ OR } Y \\
    0 & 0 & 0 \\
    0 & 1 & 1 \\
    1 & 0 & 1 \\
    1 & 1 & 1 \\
\end{array}
\end{aligned}\]

Sabendo que a saída foi 1 não é possível identificar qual/quais bits
eram 1.

\begin{itemize}
\tightlist
\item
  \textbf{Quântica}: seus circuitos são reversíveis, isso ocorre, pois,
  seus operadores são unitários.
\end{itemize}

\textbf{Observação}

Embora a evolução temporal seja reversível durante o processamento da
informação no circuito quântico, a medição dos qubits é um processo
irreversível.

\section{Portas Lógicas Quânticas}\label{sec:portas-quanticas}

As portas lógicas quânticas são operações {unitárias} que ao atuar em um
estado inicial levam para outro estado final, ou seja, funcionam como
rotações na esfera de Bloch.
A seguir, alguns exemplos de portas lógicas quânticas que atuam sobre um qubit.

\subsection{Porta X}\label{subsec:porta-x}

Essa porta é o equivalente a porta NOT da computação clássica.

Matriz

\[\begin{aligned}
X = \sigma_x =
\begin{bmatrix}
    0 & 1 \\
    1 & 0
\end{bmatrix}
\end{aligned}\]

Comportamento

\[\begin{aligned}
\begin{matrix}
    X \left| 0 \right\rangle &=& \left| 1 \right\rangle \\
    X \left| 1 \right\rangle &=& \left| 0 \right\rangle
\end{matrix}
\end{aligned}\]

Símbolo

\begin{figure}[!htp]
    \centering
    \includesvg[width=0.15\textwidth,height=\textheight]{utils/gates/xgate.svg}
    \caption{Representação da porta X.}
    \label{fig:xgate}
\end{figure}

ou ainda

\begin{figure}[!htp]
    \centering
    \includesvg[width=0.15\textwidth,height=\textheight]{utils/gates/targgate.svg}
    \caption{Outra representação da porta X.}
    \label{fig:targ-gate}
\end{figure}


\subsection{Porta Y}\label{subsec:porta-y}

Matriz

\[\begin{aligned}
Y = \sigma_y =
\begin{bmatrix}
    0 & -i \\
    i & 0
\end{bmatrix}
\end{aligned}\]

Comportamento

\[\begin{aligned}
\begin{matrix}
    Y \left| 0 \right\rangle &=& i\left| 1 \right\rangle \\
    Y \left| 1 \right\rangle &=& -i\left| 0 \right\rangle
\end{matrix}
\end{aligned}\]

Símbolo

\begin{figure}[!htp]
    \centering
    \includesvg[width=0.15\textwidth,height=\textheight]{utils/gates/ygate.svg}
    \caption{Representação da porta Y.}
    \label{fig:ygate}
\end{figure}


\subsection{Porta Z}\label{subsec:porta-z}

A porta Z introduz uma fase relativa de \(\pi\) entre os estados da base
computacional.

Matriz

\[\begin{aligned}
Z = \sigma_z =
\begin{bmatrix}
    1 & 0 \\
    0 & -1
\end{bmatrix}
\end{aligned}\]

Comportamento

\[\begin{aligned}
\begin{matrix}
    Z \left| 0 \right\rangle &=& \left| 0 \right\rangle \\
    Z \left| 1 \right\rangle &=& -\left| 1 \right\rangle
\end{matrix}
\end{aligned}\]

Símbolo

\begin{figure}[!htp]
    \includesvg[width=0.15\textwidth,height=\textheight]{utils/gates/zgate.svg}
    \caption{Representação da porta Z.}
    \label{fig:zgate}
\end{figure}

\subsection{Porta Hadamard}\label{subsec:porta-hadamard}

Essa porta gera uma superposição dos estados da base computacional.

Matriz

\[\begin{aligned}
H = \dfrac{1}{\sqrt{2}}
\begin{bmatrix}
    1 & 1 \\
    1 & -1
\end{bmatrix}
\end{aligned}\]

Comportamento

\[\begin{aligned}
\begin{matrix}
    H \left| 0 \right\rangle &=& \dfrac{1}{\sqrt{2}} \left( \left| 0 \right\rangle + \left| 1 \right\rangle \right) &=& \left| + \right\rangle \\
    H \left| 1 \right\rangle &=& \dfrac{1}{\sqrt{2}} \left( \left| 0 \right\rangle - \left| 1 \right\rangle \right) &=& \left| - \right\rangle \\
\end{matrix}
\end{aligned}\]

Símbolo

\begin{figure}[!htp]
    \centering
    \includesvg[width=0.15\textwidth,height=\textheight]{utils/gates/hgate.svg}
    \caption{Representação da porta H.}
    \label{fig:hgate}
\end{figure}


\subsection{Portas Controladas}\label{subsec:portas-controladas}

Para se fazer computação quântica universal, ou seja, realizar todas as
transformações unitárias desejadas entre os qubits de entrada e saída em
um algoritmo, é necessário realizar operações que façam dois ou mais
qubits interagirem entre si.
Tais portas podem envolver um qubit de controle e o outro como alvo, sendo possível generalizá-la para
múltiplos qubits de controle e de alvo.
Segue o exemplo para porta controlada X, ou CNOT, com um controle e um alvo.

Matriz

\[\begin{aligned}
\text{CNOT} =
\begin{bmatrix}
    1 & 0 & 0 & 0 \\
    0 & 1 & 0 & 0 \\
    0 & 0 & 0 & 1 \\
    0 & 0 & 1 & 0
\end{bmatrix}
\end{aligned}\]

Comportamento

\[\begin{aligned}
\begin{matrix}
    \text{CNOT} \left| 00 \right\rangle &=& \left| 00 \right\rangle \\
    \text{CNOT} \left| 01 \right\rangle &=& \left| 01 \right\rangle \\
    \text{CNOT} \left| 10 \right\rangle &=& \left| 11 \right\rangle \\
    \text{CNOT} \left| 11 \right\rangle &=& \left| 10 \right\rangle
\end{matrix}
\end{aligned}\]

Símbolo

\begin{figure}[!htp]
    \centering
    \includesvg[width=0.15\textwidth,height=\textheight]{utils/gates/cxgate.svg}
    \caption{Representação da porta X-controlada.}
    \label{fig:cnot}
\end{figure}

ou ainda

\begin{figure}[!htp]
    \centering
    \includesvg[width=0.15\textwidth,height=\textheight]{utils/gates/ctarggate.svg}
    \caption{Outra representação da porta X-controlada.}
    \label{fig:cnot2}
\end{figure}


\section{Emaranhamento}\label{sec:emaranhamento}

Estados emaranhados são aqueles que não podem ser escritos como produto
tensorial de estados de 1 qubit, ou seja, não é possível separá-los.
Os mais conhecidos são os estados de Bell, os quais envolvem apenas 2
qubits, sendo dados por:

\[\begin{aligned}
\begin{matrix}
\left| \beta_{00} \right\rangle &=& \left| \Phi^+ \right\rangle &=& \dfrac{1}{\sqrt{2}} \left( \left| 00 \right\rangle + \left| 11 \right\rangle \right) \\
\left| \beta_{01} \right\rangle &=& \left| \Phi^- \right\rangle &=& \dfrac{1}{\sqrt{2}} \left( \left| 00 \right\rangle - \left| 11 \right\rangle \right) \\
\left| \beta_{10} \right\rangle &=& \left| \Psi^+ \right\rangle &=& \dfrac{1}{\sqrt{2}} \left( \left| 01 \right\rangle + \left| 10 \right\rangle \right) \\
\left| \beta_{11} \right\rangle &=& \left| \Psi^- \right\rangle &=& \dfrac{1}{\sqrt{2}} \left( \left| 01 \right\rangle - \left| 10 \right\rangle \right)
\end{matrix}
\end{aligned}\]

Os estados emaranhados são apontados como sendo os responsáveis por fazer
não apenas a computação quântica mais veloz do que a computação
clássica, mas também permitem aumentar a precisão de medidas de
observáveis físicos e realizar comunicação de forma segura.

\subsection{Criando um Estado de Bell}\label{subsec:criando-um-estado-de-bell}

Com o conceito de emaranhamento explicado, resta saber como criá-lo.
Como exemplo, o estado \(\left| \beta_{00} \right\rangle\) será criado, na \autoref{fig:estados-bell} temos o circuito para isso e em \autoref{lst:estado-bell} o código para o mesmo usando Ket.

\begin{figure}
    \centering
    \includesvg[width=0.4\textwidth,height=\textheight]{utils/bell_state.svg}
    \caption{Circuito para criar um estado de Bell.}
    \label{fig:estados-bell}
\end{figure}

\begin{listing}[!htb]
\begin{minted}{python}
q0, q1 = quant(2)   # cria dois qubits
H(q0)               # aplica a porta de Hadamard no qubit 0
ctrl(q0, X, q1)     # aplica a porta X no qubit 1, com o qubit 0 como controle
\end{minted}
\caption{Criando um estado de Bell em Ket.}
\label{lst:estado-bell}
\end{listing}

Seja \(\left| \psi \right\rangle = q_0 \otimes q_1\).
Após a aplicação da porta de Hadamard, teremos
\(q_0 = \dfrac{1}{\sqrt{2}} \left( \left| 0 \right\rangle + \left| 1 \right\rangle \right)\),
conforme visto anteriormente.
Logo,

\[\begin{aligned}
\begin{matrix}
    \left| \psi \right\rangle &=&    \dfrac{1}{\sqrt{2}} \left( \left| 0 \right\rangle + \left| 1 \right\rangle \right) \otimes \left| 0 \right\rangle \\
    &=& \dfrac{1}{\sqrt{2}} \left( \left| 00 \right\rangle + \left| 10 \right\rangle\right)
\end{matrix}
\end{aligned}\]

Na sequência, temos uma porta CNOT, com o qubit 0 como controle e o qubit 1 como alvo.
Gerando a seguinte situação

\[\begin{aligned}
\begin{matrix}
    \left| \psi \right\rangle &=&        \text{CNOT} \left[ \dfrac{1}{\sqrt{2}} \left( \left| 00 \right\rangle + \left| 10 \right\rangle\right) \right] \\
    &=& \dfrac{1}{\sqrt{2}} \left( \text{CNOT} \left| 00 \right\rangle + \text{CNOT} \left| 10 \right\rangle \right) \\
    &=& \dfrac{1}{\sqrt{2}} \left( \left| 00 \right\rangle + \left| 11 \right\rangle \right) \\
    &=& \left| \beta_{00} \right\rangle
\end{matrix}
\end{aligned}\]

Portanto, com apenas duas portas é possível gerar uma situação de
emaranhamento.

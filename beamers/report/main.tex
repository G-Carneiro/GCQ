%! Author = gabriel
%! Date = 9/17/22

% Preamble
\documentclass[
    12pt,
    xcolor={svgnames},
%    notes
]{beamer}

% Packages
\usepackage{amsmath}
\usepackage{pgfplots}
\usepackage{minted}
\usepackage[utf8]{inputenc}
\usepackage[T1]{fontenc}                        % Pacote de fontes
\usepackage[brazil]{babel}                      % Tradução para português (pt-br)
\usepackage{hyperref}
\usepackage{tikz}
\usepackage{braket}
\usepackage{svg}
\usepackage{pgfpages}

\usetheme{Madrid}
\usecolortheme[named=darkgray]{structure}
\setbeamercolor{background canvas}{bg=gray}
%\setbeamercolor{frametitle}{bg=darkgray}
\setbeamercolor{footline}{bg=darkgray}
\setbeamertemplate{enumerate items}[default]
\setbeamertemplate{itemize item}[circle]
\setbeamertemplate{itemize subitem}[circle]
%\setbeamercolor{titlelike}{fg=white}
%\usefonttheme{structuresmallcapsserif}
\usefonttheme{serif}
\title{Qiskit: aprendendo a programar computadores quânticos}
\author{\href{https://github.com/G-Carneiro}{Gabriel Carneiro \\
Orientador: Eduardo Inácio Duzzioni}}
%\institute[UFSC]{Universidade Federal de Santa Catarina}
\date{10 de setembro de 2022}

\setminted[python]{
    breaklines,
    fontsize=\scriptsize,
    linenos
}

\titlegraphic{
    \includegraphics[scale=0.15]{../images/logos/vertical_sigla_PB_fundo_claro}
}

\setbeamertemplate{title page}[default][rounded=true]
\setbeamertemplate{footline}{
    \leavevmode%
    \hbox{%
        \begin{beamercolorbox}[wd=0.7\paperwidth,ht=2.25ex,dp=1ex,leftskip=1em]{footline}%
            \includegraphics[scale=0.05]{../images/logos/vertical_sigla_PB_fundo_claro}\hspace*{1em}%
            \usebeamerfont{author in head/foot}\href{https://github.com/G-Carneiro}{Gabriel Carneiro}
        \end{beamercolorbox}%
%        \begin{beamercolorbox}[wd=.3\paperwidth,ht=2.25ex,dp=1ex,center]{title in head/foot}%
%            \usebeamerfont{title in head/foot}\insertshorttitle
%        \end{beamercolorbox}%
        \begin{beamercolorbox}[wd=.3\paperwidth,ht=2.25ex,dp=1ex,right]{date in head/foot}%
%            \usebeamerfont{date in head/foot}\insertshortdate{}\hspace*{2em}
            \insertframenumber{} / \inserttotalframenumber\hspace*{2ex}
        \end{beamercolorbox}}%
    \vskip0pt%
}

% Document
\begin{document}
    \begin{frame}[plain]
        \titlepage
    \end{frame}

    \begin{frame}{Motivação}
        \begin{itemize}
            \item Crescimento
            \item Eficiência
            \item Mercado
        \end{itemize}
    \end{frame}

    \begin{frame}{Computação Quântica}
        \begin{columns}
            \begin{column}{0.5\textwidth}
            \begin{itemize}
                \item Revisão de Algebra Linear
                \item Circuito Quântico
                    \begin{itemize}
                        \item Qubit
                        \item Portas
                        \item Superposição
                        \item Emaranhamento
                    \end{itemize}
                \item Qiskit
                \item Ket
                    \begin{itemize}
                        \item QuBOX
                    \end{itemize}
            \end{itemize}
            \end{column}
            \begin{column}{0.5\textwidth}
                \begin{figure}
                    \includegraphics[scale=0.2]{../images/figures/qubox}
                    \caption{Retirado de https://qubox.ufsc.br}
                    \label{fig:qubox}
                \end{figure}
            \end{column}
        \end{columns}
    \end{frame}

    \begin{frame}{Algoritmos}
        \begin{itemize}
            \item Busca de Grover
            \item Transformada de Fourier Quântica
            \item Estimativa de Fase
            \item Busca de Ordem
            \item Fatoração de Shor
        \end{itemize}
    \end{frame}

    \begin{frame}{Grover - Tempo de Execução}
        \begin{figure}
        \label{fig:figure}
        \centering
        \begin{tikzpicture}
            \begin{axis}[
                xmin=2,
                ymin=0,
                ymax=10,
                xlabel={Nº Qubits},
                ylabel={Tempo (s)},
                legend pos=north west,
                legend entries={qubox-pbw, ket-local, qubox-pbwd, qiskit-aer}]
                \addplot[DarkBlue, mark=diamond*] table [x=qubits, y=average] {../grover/dat/ket_qubox_01.dat};
                \addplot[DarkRed, mark=*] table [x=qubits, y=average] {../grover/dat/ket_local_01.dat};
                \addplot[DarkGreen, mark=square*] table [x=qubits, y=average] {../grover/dat/ket_qubox_pbwd_01.dat};
                \addplot[Purple, mark=triangle*] table [x=qubits, y=average] {../grover/dat/qiskit_aer_01.dat};
            \end{axis}
        \end{tikzpicture}
        \end{figure}
    \end{frame}

    \begin{frame}{Conclusão}
        \begin{itemize}
            \item Área recente
            \item Preparação
            \item Trabalhos futuros
        \end{itemize}
    \end{frame}
\end{document}